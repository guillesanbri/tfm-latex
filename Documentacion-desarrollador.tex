\apendice{Documentación del Desarrollador}

Este apéndice presentará el plan de proyecto elaborado para la realización del trabajo. En el caso de trabajos que supongan el desarrollo de software, será sustituído por el Plan de Desarrollo de Software.

\section{Introducción}
El siguiente apéndice se compone de tres apartados principales: En el primero, se discutirá la estructura de directorios en la que se organiza el proyecto, especificando los contenidos de cada uno; el segundo apartado, por otro lado, documenta los pasos que es necesario llevar a cabo para instalar los componentes \textit{software} necesarios para ejecutar los modelos y programas desarrollados en este trabajo, además de incluir las instrucciones de descargo de los datos necesarios para entrenar los modelos; por último, en el tercer apartado, se especifican los comandos necesarios para ejecutar cada uno de los \textit{scripts} disponibles, así como una descripción de los resultados de su ejecución.

\section{Estructura de directorios}
Lorem ipsum

\section{Instalación de \textit{software} y descarga de datos}
A continuación se enumeran los pasos necesarios para instalar el \textit{software} de los entornos de desarrollo del proyecto y para descargar los datos empleados durante el proyecto. Cabe destacar que \textbf{parte de estas instrucciones son específicas para equipos con sistema operativo Ubuntu 20.04 LTS}, por lo que se proporcionan enlaces con información relevante para otras distribuciones. Además, se presupone que el equipo en el que se va a llevar a cabo el proceso de instalación tiene una tarjeta gráfica NVIDIA con los controladores gráficos (\textit{drivers}) correspondientes correctamente instalados.

\subsection{Docker Engine}
Para poder disfrutar de las ventajas de los contenedores Docker, es necesario instalar el Docker Engine. Tal y como se ha mencionado previamente, estas instrucciones son para sistemas con Ubuntu. En el siguiente enlace (\url{https://docs.docker.com/engine/install/}), hay disponibles instrucciones para otras distribuciones.

\begin{enumerate}
\item Eliminamos versiones anteriores de Docker (en caso de existir):

\texttt{\$ sudo apt-get remove docker docker-engine docker.io containerd runc}

\item Instalamos paquetes necesarios para usar un repositorio a través de HTTPS:

\texttt{\$ sudo apt-get update}

\texttt{\$ sudo apt-get install apt-transport-https ca-certificates curl \textbackslash \\ gnupg lsb-release}

\item Añadimos la clave GPG de Docker:

\texttt{\$ curl -fsSL https://download.docker.com/linux/ubuntu/gpg | sudo gpg \textbackslash \\ --dearmor -o /usr/share/keyrings/docker-archive-keyring.gpg}

\item Configuramos el repositorio de Docker:

\texttt{\$ echo \textbackslash \\ 
"deb [arch=amd64 signed-by=/usr/share/keyrings/docker-archive-keyring.gpg] \textbackslash \\ 
https://download.docker.com/linux/ubuntu \$(lsb{\_}release -cs) stable"{ }| \textbackslash \\ sudo tee /etc/apt/sources.list.d/docker.list >{ }/dev/null}

\item Instalamos Docker:

\texttt{\$ sudo apt-get update}

\texttt{\$ sudo apt-get install docker-ce docker-ce-cli containerd.io}

\item Por último, comprobamos que la instalación se haya llevado a cabo correctamente:

\texttt{\$ sudo docker run hello-world}

% \par\noindent\rule{0.965\textwidth}{0.4pt}

\end{enumerate}

En este punto, Docker ya está instalado, pero es necesario ejecutarlo como superusuario. Para remediar esto (opcional), es posible crear un grupo \texttt{"docker"} y añadir nuestro usuario. Al hacer esto, \textbf{pueden aparecer riesgos de seguridad} en función del sistema en el que se esté llevando a cabo la instalación. Se puede encontrar más información en:

\begin{itemize}

\item \url{https://docs.docker.com/engine/install/linux-postinstall/}

\item \url{https://docs.docker.com/engine/security/#docker-daemon-attack-surface}

\end{itemize}


\begin{enumerate}

\item Añadimos un grupo \texttt{"docker"}:

\texttt{\$ sudo groupadd docker}

\item Añadimos nuestro usuario al grupo creado:

\texttt{\$ sudo usermod -aG docker \$USER}

\item Salimos y volvemos a iniciar la sesión para re-evaluar la pertenencia a grupos. En sistemas Linux también se puede ejecutar \texttt{newgrp docker}. 

\item Ahora es posible ejecutar el contenedor \texttt{hello-world} sin usar \texttt{sudo}.

\texttt{\$ docker run hello-world}

% \texttt{\$ sudo chown "\$USER":"\$USER"{}{ }/home/"\$USER"{}/.docker -R}
% \texttt{\$ sudo chmod g+rwx "\$HOME/.docker"{} -R}

\end{enumerate}

Por otro lado, es posible configurar el servicio de Docker para que se inicie automáticamente al encender el sistema a través de \texttt{systemctl}:

\begin{itemize}

\item \texttt{\$ sudo systemctl enable docker.service}

\item \texttt{\$ sudo systemctl enable containerd.service}

\end{itemize}

\subsection{NVIDIA Container Toolkit}
Una vez instalado el Docker Engine, es necesario instalar el NVIDIA Container Toolkit, que envuelve la instalación anterior y traslada las primitivas de CUDA desde el CUDA instalado dentro de los contenedores al driver de la GPU en el equipo donde se está ejecutando el contenedor. Estas instrucciones son para equipos con distribuciones Ubuntu y Debian. Para los pasos correspondientes en otras distribuciones, consultar la siguiente url: \url{https://docs.nvidia.com/datacenter/cloud-native/container-toolkit/install-guide.html#getting-started}

\begin{enumerate}
\item Configuramos el repositorio y copiamos la clave GPG:

\texttt{\$ distribution=\$(. /etc/os-release;echo \$ID\$VERSION{\_}ID) \textbackslash \\ \&\& curl -s -L https://nvidia.github.io/nvidia-docker/gpgkey | \textbackslash \\ sudo apt-key add - \textbackslash \\ \&\& curl -s -L \textbackslash \\ https://nvidia.github.io/nvidia-docker/\$distribution/nvidia-docker.list | \textbackslash \\ sudo tee /etc/apt/sources.list.d/nvidia-docker.list}

\item Instalamos el paquete correspondiente:

\texttt{\$ sudo apt-get update}

\texttt{\$ sudo apt-get install -y nvidia-docker2}

\item Para terminar la instalación, reiniciamos el servicio de Docker:

\texttt{\$ sudo systemctl restart docker}

\item Para comprobar que la instalación es correcta, podemos ejecutar un contenedor de una imagen con CUDA instalado:

\texttt{\$ docker run --rm --gpus all nvidia/cuda:11.0-base nvidia-smi}

\end{enumerate}

Lorem ipsum: nvidia-smi y comentario sobre el driver

\subsection{Construcción de la imagen de Docker}

Lorem ipsum: Contar como generar la imagen de Docker

\subsection{Instalación de wandb (Opcional)}

Lorem ipsum: Decir que es opcional, que se puede trabajar con el wandb que se ha instalado dentro del contenedor, pero que por comodidad, se instala en local para manejar los sweeps y poder pasar las credenciales facilmente a los contenedores docker.

\section{Ejecución}
